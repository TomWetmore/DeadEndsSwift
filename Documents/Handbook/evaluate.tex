\section{EVALUATE}
\texttt{ProgramNode}s represent statements and expressions. The functions in
Evaluate.swift evaluate the six types of \texttt{ProgramNode}s that
hold expressions.

The \texttt{evaluate} and \texttt{...evaluate} functions it calls are all methods
on the \texttt{Program} class. This gives
them access to the static and dynamic state of the \texttt{Program}
that is being interpreted.

Evaluation always converts a \texttt{ProgramNode} (and others it may contain)
into a \texttt{ProgramValue}.

\subsection{Evaluate}

\begin{verbatim}
extension Program
    func evaluate(PNode) throws -> ProgramValue
        case .integer(integer)... // return ProgramValue.integer(integer)
        case .double(double)...   // return ProgramValue.double(double)
        case .string(string)...   // return ProgramValue.string(string)
        case .identifier...       // call evaluateIdent
        case .builtinCall...      // call evaluateBuiltin
        case .functionCall...     // call evaluateFunction
\end{verbatim}

\texttt{evaluate} is the generic evaluator. There are six types of \texttt{ProgramNode}s
that it evaluates: .\texttt{integer}, \texttt{.float}, \texttt{.string},
\texttt{.identifier}, \texttt{.builtinCall}, and \texttt{.functionCall}.

\subsection{Secondary Evaluate Functions}

\begin{verbatim}
extension Program
    func evaluateIdent(ProgramNode) throws -> ProgramValue
    func evaluateBuiltin(ProgramNode) throws -> ProgramValue
    func evaluateFunction(ProgramNode) throws -> ProgramValue
\end{verbatim}

\texttt{evaluateIdent} evaluates an \texttt{.identifier} node by
looking up the identifier in the symbol tables. The ProgramValue of
the identifier is returned.

\texttt{evaluateBuiltin} evaluates a \texttt{.builtinCall} node.
It checks the numbers of arguments and parameters; it calls
\texttt{evaluate} on the arguments to get their values, it creates a
SymbolTable for the new frame, and then
it calls the builtin function (a Swift closure).

\texttt{evaluateFunction} evaluates a \texttt{.functionCall} node.
It checks the numbers of arguments and parameters; it calls
evaluate on the arguments to get their values; it looks up the
function in the table of of user defined functions; it creates a
SymbolTable for the new frama; and then it calls the defined function, make
of ProgramNodes.

\subsection{Support Evaluate Functions}

\begin{verbatim}
extension Program
    func evaluateCondition([ProgramNode]) throws -> Bool
    func evaluateBoolean(ProgramNode) throws -> ProgramValue
    func pvalueToBoolean(PValue) -> Bool
    func evaluatePerson(ProgramNode) throws -> GedcomNode?
    func evaluateFamily(ProgramNode) throws -> GedcomNode?
    func evaluateGedcomNode(PNode) throws -> GNode?
\end{verbatim}

\texttt{evaluateCondition} evaluates the conditions found in if
and while statements. The array can have one or two nodes. If there
is one it is evaluated, coerced to boolean and returned.
If there are two nodes the first is an .identifier. The second is
evaluated and assigned to the identifer. The second value is coerced
to boolean and returned.(

\texttt{evaluateBoolean} evaluates a ProgramNode to a \texttt{.boolean}
ProgramValue. NEEDED?

\texttt{pvalueToBoolean} coerces a ProgramValue to a boolean. NEEDED?

\texttt{evaluatePerson} evaluates a \texttt{.gnode ProgramNode} to the root
GedcomNode of a person. Several builtin functions take persons as
parameters, and this function is convenient way to evaluate those
arguments.

\texttt{evaluateFamily} evaluates a \texttt{.gnode ProgramNode} the
root GedcomNode of a family. Several builtin function take families
as parameters, and this function is a convenient way to evaluage those
arguments.

\texttt{evaluateGedcomNode} evaluates a \texttt{.gnode ProgramNode} to a
\texttt{GedcomNode}, which can have any tag or be at any level.
