\section{Import Stack -- \texttt{ImportStack.swift}}
%==================================================
%
% INTRODUCTION
%
%==================================================
\subsection{Introduction}
The \emph{import stack} consists of the functions that read Gedcom
files and create the \texttt{Array}s of \texttt{GedcomNode} trees,
one for each Gedom record found in the file.
Each record consists of a tree of
\texttt{GedcomNode}s
where the root corresponds a level 0 (e.g., \texttt{0 INDI},
\texttt{0 FAM}) line. After the records are created validation
takes place and a Database may be created. Validation is
covered in Section~\ref{sec:validation} and the Database
in Section~\ref{sec:database}.

The functions making up the import stack are each described
in a section below.

%======================================================
%
% EXTRACT FIELDS
%
%=======================================================

\subsection{\texttt{extractFields}}
\begin{verbatim}
enum ReadResult {
    case success(level: Int, key: String?, tag: String, value: String?)
    case failure(errmsg: String)

func extractFields(from line: String) -> ReadResult 
\end{verbatim}

\texttt{extractFields} extracts the level, key, tag and value from a
String that holds a single Gedcom line.
The field values are returned via a \texttt{ReadResult} value, an
enumeration with one value for successful returns and another for errors.

\texttt{extractFields} is called only by \texttt{getDataNodesFromPath}.
\subsubsection{Note on Errors}
As far as is possible errors that occur when running the input stack do not stop
 processing.
Errors accumulate so the user can be apprised of as many errors as possible.

%======================================================
%
% GET DATA NODES FROM PATH
%
%=======================================================	

\subsection{\texttt{getDataNodesFromPath}}
\begin{verbatim}
struct DataNodes<Type> { var nodes: [(GedcomNode, Type)] ... }
 
func getDataNodesFromPath(path: String, tagmap: inout TagMap, keymap: inout KeyMap,
    errlog: inout ErrorLog) -> DataNodes<Int>?
\end{verbatim}

\texttt{getDataNodesFromPath} returns all lines from a Gedcom source as
elements in a flat
\texttt{[DataNodes<Int>]} array.
Elements of the array are (\texttt{GedcomNode,~Int}) tuples, where the
\texttt{GedcomNode}
holds the fields returned by \texttt{extractFields}, and the integer holds
the the node's level -- a
\texttt{GedcomNode} does not have a level field, but the next step needs
the levels in order to build the trees.

Input parameters:

\texttt{path} -- path to a Gedcom file.

\texttt{tagmap} -- \texttt{inout TagMap} of unique tag strings. See Section ????.

\texttt{keyMap} -- \texttt{inout [String:Int]} dictionary that maps
a record's key to its starting line number in the Gedcom file.
It is used when generating error messages.

\texttt{errlog} -- \texttt{inout ErrorLog} where errors found when
processing the Gedcom file are recorded. The source is fully processed regardless of errors.
If there are errors \texttt{nil} is returned.

Summary: \texttt{getDataNodesFromPath} reads a Gedcom file and breaks it
into an array of lines.
It calls \texttt{extractFields} on each line to get its \texttt{GedcomNode}.
It then adds a (\texttt{GedcomNode}, level)
tuple to the return array.

\texttt{getDataNodesFromPath} is called by \texttt{getRecordsFromPath} as
the first step in reading the Gedcom records from a file.

%======================================================
%
% GET RECORDS FROM DATA NODES
%
%=======================================================	

\subsection{\texttt{getRecordsFromDataNodes}}
\begin{verbatim}
typealias RootList = [GedcomNode]

func getRecordsFromDataNodes(datanodes: DataNodes<Int>, keymap: KeyMap,
    errlog: inout ErrorLog) -> RootList {
\end{verbatim}

\texttt{getRecordsFromDataNodes} converts an array of
\texttt{GedcomNode}s, in the form of (\texttt{GedcomNode}, level)
pairs in a \texttt{DataNodes<Int>} array, from a Gedcom source,
into a \texttt{RootList}, the array of root
\texttt{GedcomNodes} of all records from the Gedcom file.

Because the input is a sequential list of all the \texttt{GedcomNode}s
from the file, \texttt{gedRecordsFromDataNodes} also needs the levels
of the \texttt{GedcomNode}s to be able to construct the trees.
This is why its input is a \texttt{DataNodes<Int>} object rather
than a simple \texttt{[GedcomNode]} array. It needs the levels to
guide the tree building.
It converts a flat array of all \texttt{GedcomNode}s, into an array
of root nodes with their trees attached.
It builds the trees using a simple state machine.

\texttt{getRecordsFromDataNodes} is called by
\texttt{getRecordsFromPath},
immediately after it calls \texttt{get\-Data\-Nodes\-From\-Path}.
This completes the two step process where the first gets the
full list of
\texttt{GedcomNode}s
from the file, and the second builds the \texttt{GedcomNode} trees
and returns the array of the root \texttt{GedcomNode}s.
\texttt{RootList} is an alias for \texttt{[GedcomNode]}
to be used when the array of nodes contains only roots.

%======================================================
%
% GET RECORDS FROM PATH
%
%=======================================================

\subsection{\texttt{getRecordsFromPath}}

\begin{verbatim}
func getRecordsFromPath(path: String, tagmap: inout TagMap, keymap: inout KeyMap,
    errorlog: inout ErrorLog) -> RootList?
\end{verbatim}
\texttt{getRecordsFromPath} returns the Gedcom records from a source. It uses
\texttt{getDataNodesFromPath} and \texttt{getRecordsFromDataNodes} to
create a \texttt{RootList} of records.

See the previous two function for details on the process and the
meanings of the parameters.

\texttt{getRecordsFromPath} and the functions it calls make up
the input stack.
The function that follows is located in
\texttt{InputStack.swift}, but could
have been included in the validation software.


%======================================================
%
% GET VALID RECORDS FROM PATH
%
%=======================================================

\subsection{\texttt{getValidRecordsFromPath}}
\begin{verbatim}
func getValidRecordsFromPath(path: String, tagmap: inout TagMap,
    keymap: inout KeyMap, errlog: inout ErrorLog)
    -> (index: RecordIndex, persons: RootList, families: RootList)?
\end{verbatim}
\texttt{getValidRecordsFromPath} uses the input stack by calling
\texttt{getRecordsFromPath} to get the Gedcom
records from a file, and it then validates those records by
calling:
\begin{verbatim}
checkKeysAndReferences
validatePersons
validateFamilies
\end{verbatim}

\texttt{getValidRecordsFromPath} returns a triple consisting of:

\texttt{index} -- \texttt{[String:GedcomNode]} dictionary mapping
record keys to their root \texttt{GedcomNode}s.

persons -- RootList, an Array listing all GedcomNode roots of
persons from the file.

families: RootList list all GedcomNode roots of famlies
from the file.

See the Validation Section for the documentation on validation and
these validation functions.





\subsection{Looseends}
ValidationContext is defined in ImportStack.swift, though not
used there.