\section{GedcomNode}
\subsection{Introduction}

A \texttt{GedcomNode} encodes a line from a Gedcom file.
It is a key class in DeadEnds.
Records in a
DeadEnds database are GedcomNode trees.

DeadEnds does not have records for persons, families, etc.
All DeadEnds records are references to roots of node trees.
DeadEnds records are stored in a Database.

Other genealogical systems define
specific types for genealogical entities.
Of course DeadEnds has routines that deal specifically
with persons and families, etc., 
but all their interfaces use references to \texttt{GedcomNode}s.

\subsection{Class GedcomNode}
\begin{verbatim}
public class GedcomNode: CustomStringConvertible {

    var key: String? // Key; only on root nodes.
    var tag: String  // Gedcom tag; mandatory.
    var value: String? // Value; optional.
    var nextSibling: GedcomNode? // Next sibling; optional.
    var firstChild: GedcomNode?  // First child; optional.
    weak var parent: GedcomNode? // Parent; not on root nodes.
    
    public var description: String
    init(key: String? = nil, tag: String, value: String? = nil)
    func printTree(level: Int = 0, indent: String = "")
    lazy var childrenByTag: [String: [GedcomNode]]
    static func == (lhs: GedcomNode, rhs: GedcomNode) -> Bool

\end{verbatim}
\texttt{key}, \texttt{tag} and
\texttt{value} hold the key (\emph{cross reference identifier}), tag,
and value of a Gedcom line. The line's level in not stored because it
can be computed.
\texttt{firstChild}, \texttt{nextSibling} and \texttt{parent} hold
the tree structure.

\texttt{description} returns a description of a node.

\texttt{init} initializes and returns a new node.

\texttt{printTree} is a debug method that prints the tree rooted at a node.

\texttt{childrenByTag} gets the dictionary of arrays of child nodes
indexed by a tag. It is a lazy computed property.
\subsection{Convenience Accessors}

\begin{verbatim}
extension GedcomNode
    func valueForTag(tag: String) -> String?
    func valuesForTag(tag: String) -> [String] {
    func childWithTag(tag: String) -> GedcomNode?
    func childrenWithTag(tag: String) -> [GedcomNode]
    
	
\end{verbatim}

These accessor methods give access to the children of a
\texttt{GedcomNode} with certain properties:

\texttt{valueForTag} returns the value of the first child with a given tag.

\texttt{valuesForTag} returns the array of non-nil values from a list of children with a given tag.

\texttt{childWithTag} finds the first child node a given tag.

\texttt{childrenWithTag} finds the array of children with a given tag.

\subsection{TagMap}
Gedcom files can be large; the same tags may occur thousands of times.
The TagMap class provides a way for every GedcomNode with the same tag
to share the same string.
\begin{verbatim}
class TagMap
    private var map: [String:String] = [:]
    func intern(tag: String) -> String
\end{verbatim}

\texttt{intern} returns the unique copy of a string.
\subsection{Notes and ToDo's}
How about a computed property for level?

Contrast the DeadEnds way of encoding all information in trees of
\texttt{GedcomNode}s, rather than as specialized records with specialized
types for things like names, events, dates, places, relationships.



