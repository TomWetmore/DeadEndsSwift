\section{SEXPRESSION}

\subsection{Introduction}

\texttt{SExpression}s transfer DeadEnds programs between executables.
A LifeLines-based program parses LifeLines programs and
writes them as \texttt{SExpressions} to files. DeadEnds 
parses the files and uses the expressions to build \texttt{ProgramNode}
trees that are interpreted by the interpreter
component.

It would be possible to interpret SExpressions's directly; this may
be a future enhancement.

\subsection{enum SExpression}

\begin{verbatim}
public enum SExpression {
    case atom(String, line: Int?) // Atoms (strings and optional line number).
    case list([SExpression])  // Lists (nested expressions).

\end{verbatim}
An \texttt{SExpression} is an enumeration with two cases. The first is \texttt{.atom},
a string with optional integer, and the second is .list, an array
of SExpressions. The optional integer ties .atoms back
to their line numbers in the original script.

\subsection{SExpressionParser}
\begin{verbatim}
public struct SExpressionParser {

    public init(String)
    public mutating func parseProgramSExpression() throws -> SExpression
    
    private mutating func parseSExpression(Int = 0) throws -> SExpression
    private static func tokenize(String) -> [String]
    private static func splitAtomAndLine(String) -> (String, Int?)
    public func tokenArray() ->  [String] // debug

\end{verbatim}

\texttt{SExpressionParser} is a struct whose methods parse SExpressions. The initializer tokenizes an
SExpression by calling \texttt{tokenize}. \texttt{parseProgramSExpression}
parses the tokens and returns the program-level SExpression.
\texttt{tokenize} uses a long and undecipherable regular expression; I want to
convert it to a hand-crafted lexer. \texttt{splitAtomAndLine} returns a
\texttt{(String, Int?)} tuple with an .atom's value.

\texttt{parseProgramSExpression} parses and returns the program level SExpression and checks
there is nothing left over. This SExpression is the \texttt{.list} of the procedure and function
definitions and the global declarations that make up the program.
This is checked in the \texttt{convertToTables} function.

\texttt{parseSExpression} is the general purpose, recursive SExpression parser.
It keeps track of the recursive depth of the parse, though it
is not used.


\subsection{convertToTables}

\begin{verbatim}
public func convertToTables(SExpression) throws -> (procs, funcs, globals)
\end{verbatim}

\texttt{convertToProgram} loads the program SExpression created
by the parser into dictionaries of ProgramNodes.
The program SExpression is a \texttt{.list} whose elements are procedure definitions,
function definitions, and/or global variable declarations (in any order).
This function finds those elements, creates their ProgramNode equivalents, and builds a
dictionary for each type (.procedureDef, .functionDef, .globalDeclaration).
All ProgramNodes are created via calls to the \texttt{toProgramNode} routines made
in this function. 

\subsection{SExpression extension for toProgramNode method}

\begin{verbatim}
extension SExpression {
    public func toProgramNode() throws -> ProgramNode
}
\end{verbatim}

\texttt{toProgramNode} creates a ProgramNode from an SExpression. Conversion
to ProgramNodes based on \texttt{.atoms} is done\texttt{} in this method.
Conversion to ProgramNodes from \texttt{.lists} is done by functions called from
here.

\texttt{.integer} and \texttt{.float} ProgramNodes are created from \texttt{.atoms} whose string are numeric.
\texttt{.string} ProgramNodes are created from \texttt{.atoms} whose strings are surrounded by
quotes, and \texttt{.identifier} ProgramNodes from all other .atoms.

\texttt{.list} SExpressions represent statement types and blocks. The
first \texttt{.atom} in each \texttt{.list} identifies the statement type, and there is a
function for each.

\subsection{Statement toProgramNode Functions}
\begin{verbatim}
func ifToProgramNode([SExpression]) throws -> ProgramNode
func whileToProgramNode([SExpression]) throws -> ProgramNode
func conditionToProgramNode(SExpression) throws -> [ProgramNode]
func returnToProgramNode([SExpression]) throws -> ProgramNode
func builtInCallToProgramNode([SExpression]) throws -> ProgramNode
func procedureCallToProgramNode([SExpression]) throws -> ProgramNode
func functionCallToProgramNode([SExpression]) throws -> ProgramNode
... many more to come ...
\end{verbatim}

While \texttt{toProgramNode} is a method on \texttt{SExpression}, the remainder of the
of the \texttt{...toProgramNode} routines are functions.

\texttt{ifToProgramNode} and \texttt{whileToProgramNode} create
\texttt{ProgramNodes} for \texttt{if} and \texttt{while} statements.
They call \texttt{conditionToProgramNode} to create the
\texttt{ProgramNodes} of their conditions.
\texttt{ifToProgramNode} calls \texttt{toProgramNode} on the \emph{then}
and \emph{else} clauses, and
\texttt{whileToProgramNode} calls \texttt{toProgramNode} on the loop body.
\texttt{conditionToProgramNode} distinguishes between one and two
element \texttt{.lists}.

\texttt{returnToProgramNode} returns a \texttt{.returnState}
\texttt{ProgramNode} with an optional
\texttt{ProgramNode} for the return expression.

\texttt{builtInCallToProgramNode} handles \texttt{SExpressions} for
bulitin function calls. It gets the builtin's name as a string,
creates its arguments as a list \texttt{ProgramNodes} and returns
a \texttt{.builtinCall} \texttt{ProgramNode}

\texttt{procedureCallToProgramNode} handles \texttt{SExpressions} for
calls to user defined procedures. It gets the procedure's name as a string,
creates its arguments as a list of \texttt{ProgramsNodes} and returns
a \texttt{.procedureCall} \texttt{ProgramNode}.

\texttt{funtionCallToProgramNode} handles \texttt{SExpressions} for
calls to user defined functions. It gets the function's name as a string,
creates its arguments as a list of \texttt{ProgramsNodes} and returns
a \texttt{.functionCall} \texttt{ProgramNode}.
    
\subsection{Procedure and Function Definitions}
\begin{verbatim}
func procedureDefToProgramNode(SExpression) throws -> ProgramNode
func functionDefToProgramNode(SExpression) throws -> ProgramNode 
\end{verbatim}

The \texttt{SExpression}s passed to \texttt{procedureDefToProgramNode} and
\texttt{functionDefToProgramNode} are four-element \texttt{.lists}.
The first element is the \texttt{.atom} that distinguishes procedures from functions. The
second is an \texttt{.atom} with the name.
The third is a \texttt{.list} of \texttt{.atom}
identifiers with parameter names. The fourth is a \texttt{.list} with the
body.

 \texttt{procedureDefToProgramNode} handles \texttt{SExpressions}
 holding user defined procedure definitions.
 It gets the procedure's name as a string, creates its parameters as an array
 of strings, and creates a \texttt{.block} \texttt{ProgramNode} for the body.
 It returns a \texttt{.procedureDef} \texttt{ProgramNode}.\
 

\texttt{functioneDefToProgramNode} handles \texttt{SExpressions}
holding user defined function definitions.
It gets the function's name as a string, creates its parameters as an array
of strings, and creates a \texttt{.block} \texttt{ProgramNode} from the body.
It returns a \texttt{.functionDef} \texttt{ProgramNode}.

\subsection{Utilities}
\begin{verbatim}
func isIdentifier(String) -> String?
\end{verbatim}

\texttt{isIdentifier} checks whether a string is an identifier.
